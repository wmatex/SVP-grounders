\documentclass[12pt,oneside]{article}

\usepackage[utf8]{inputenc}
\usepackage{amssymb}
\usepackage{float}
\usepackage{minted}
\usepackage{amsmath}
\usepackage{mathtools}

\DeclarePairedDelimiter{\floor}{\lfloor}{\rfloor}

\begin{document}

\title{Software project \\
\large Efficient grounding of hierarchical logical templates}

\author{Matěj Vavřinec}
\date{\today}

\maketitle

\begin{abstract}

\end{abstract}


\section{Introduction}


\section{Preliminaries}
\subsection{First-order logic}
\label{section:preliminaries:logic}
We consider a function-free first-order logic, with formulas created from set of constants,
variables, $n$-ary predicates ($n \in \mathbb{N}$) and propositional connectives $\wedge, \vee, $. A
\textit{term} is a constant or variable. An \textit{atom} is a $n$-ary predicate for some $n \in
\mathbb{N}$ applied to $n$ terms. A \textit{ground atom} is an atom, where each term consists only
of constants. A \textit{literal} is either a term or a negation of a term. A \textit{clause} is a
disjuntion of literals. A clause is called \textit{definite} (\textit{rule}) if it contains exactly
one positive literal. To increase readability a rule $h \vee \neg b_1 \vee \neg b_2 \vee \dots \vee
\neg b_n$ will be often written as $h \leftarrow b_1 \wedge b_2 \wedge \dots \wedge b_n$ as it is
often used in logic programs. The literal $h$ is referred to as \textit{head} and the conjunction of
literals as \textit{body}. Also to avoid confusion constants will be written in lowercase letters
(e.g. \textit{anne}) while variables with first letter capitalized (e.g. \textit{User}). A set of
constants and rules is a \textit{logic template}. The \textit{grounded template} is a template
consisting only of ground atoms.

\textbf{TODO:} define Herbrand model

\section{Generation}
To be able to evaluate the efficiency of a grounder, one must have a logic template with given set
of parameters influencing the grounding process in different ways. Some examples as such parameters
are number of facts, number of relations, arity of the predicates in each rule. As these templates
don't have to have any meaning or relation to any "real world" analogy, they can be synthetically
generated using a generator. The generation process is split into two parts: generation of the
dataset and generation of the hierarchical rules for a given dataset.
\subsection{Dataset generation}
A dataset is a set of constants and ground atoms. Another way of describing a
dataset is a set of SQL tables. The generator has a set of options relating to
different parameters of the resulting generated dataset. These parameters are:

\begin{itemize}
    \item \textbf{facts}: The number of different possible truth values for each predicate. More intuitively the number of rows in each table with the SQL analogy.

    \item \textbf{tables}: Number of different predicates.

    \item \textbf{relations}: Number of connections between tables. When viewing the individual tables (predicates) as nodes of multi-graph $G$, the number of relations refers to number of edges in $G$.

    \item \textbf{minimum/maximum number of columns}: The number of columns refers to the arity of a single predicate (or the number of columns of the table representing the predicate). Each table $i$ where $i \in 1 \dots T$, where $T$ is the tables parameter, will have number of columns uniformly distributed between the minimum and maximum number of columns parameters.
\end{itemize}

Listing \ref{listing:dataset:1} shows a definition of a dataset in the datalog
format and the listing \ref{listing:dataset:2} shows the same dataset in the SQL format.

\begin{listing}
    \begin{minted}[frame=leftline]{prolog}
a(a0).
a(a1).
a(a2).
b(b0, a1).
b(b1, a0).
b(b2, a1).
    \end{minted}
\caption{Example of a dataset in the datalog format. The parameters used for
generating the dataset
were: 3 facts, 2 tables, 1 relation and minimum and maximum number of columns set to
1.}
\label{listing:dataset:1}
\end{listing}

\begin{listing}
    \begin{minted}[frame=leftline]{SQL}
DROP TABLE IF EXISTS "a";
CREATE TABLE "a" (
  a_col_0 character varying(255) NOT NULL,
  PRIMARY KEY (a_col_0)
);

 %INSERT INTO "a" VALUES ('a0'),('a1'),('a2');

DROP TABLE IF EXISTS "b";
CREATE TABLE "b" (
  b_col_0 character varying(255) NOT NULL,
  b_col_1 character varying(255) NOT NULL,
  PRIMARY KEY (b_col_0)
);

INSERT INTO "b" VALUES
('b0','a1'),('b1','a0'),('b2','a1');

ALTER TABLE "b" ADD CONSTRAINT cons_b_b_col_0
FOREIGN KEY (b_col_1) REFERENCES "a" (a_col_0);
    \end{minted}
\caption{The same dataset as in \ref{listing:dataset:1}, but in SQL format.}
\label{listing:dataset:2}

\end{listing}


\subsection{Rules generation}
The set of rules is generated independently of the dataset, so that each dataset can have different
set of rules and also the rules can be generated to any given dataset, not only created using the
generator above but also any other datalog or SQL dataset created in the given format. The generator
also has number of parameters influencing the properties of the generated rules.

The generated rules are in the
format described in section \ref{section:preliminaries:logic}: $h \leftarrow d_1 \vee d_2 \vee
\dots d_n \vee r_1 \vee r_2 \vee \dots \vee d_m$, where $h$ is the head of the rule and the $d_i$,
$i \in N$, are atoms from the given dataset and $r_j$, $j \in M$, are head atoms of previous
generated rules. The generated rules are not recursive.

\textbf{TODO:} define more the rules, variables, hierarchy, process of generation.

These parameters of the controlling the genertion are following:

\begin{itemize}
 \item \textbf{base-rules}: Number of generated rules with just atoms from the given dataset.

 \item \textbf{count}: Number of generated rules with dataset and rule atoms in body.

 \item \textbf{width}: Number of atoms in the bodies of rules.

 \item \textbf{rule-proportion} $\in <0, 1>$: The proportion of rule atoms vs dataset atoms in the
body of generated rules. The number of rule-atoms in the rule body, $n_r = \floor{r \times W}$,
where $W$ is the width parameter and the number of table-atoms is $W - n_r$.

 \item \textbf{weight} $\in <0, 1>$: Weight of the table-atom with the least number of columns used
in the random atom selection. $1$ means that solely this atom will be picked and $\frac{1}{2}$
means uniform probability distribution of the table-atoms. The function is symetric.

 \item \textbf{all}: If true, that all variables from the body of the rule will also be included in
the head. If false that only the relation variables are included.

 \item \textbf{duplicity}: Number of duplicit body atoms, e.g. $2$ means, that each body-atom will
be included twice in the body.

 \item \textbf{unique}: If true that each duplicit body-atom will have unique set of variables.
\end{itemize}



Listing \ref{listing:rules:1} shows set of generated rules in the datalog format and listing
\ref{listing:rules:2} show the same set in the SQL format.

\begin{listing}
 \begin{minted}[frame=leftline]{prolog}
rule_a(A0, B0) :- a(A0), b(B0, A0).
rule_b(A0, B0) :- b(B0, A0), a(A0), rule_a(A0, B0).
 \end{minted}
\caption{Example of generated rules in the datalog format. The generation parameters were the
following: 1 base rule, 1 additional rule (\textit{count}), width = 3, rule-proportion = 0.3,
weight = 0.5, duplicity = 0.}
\label{listing:rules:1}
\end{listing}

\begin{listing}
 \begin{minted}[frame=leftline]{SQL}
CREATE OR REPLACE TEMPORARY VIEW rule_a AS
SELECT DISTINCT "a".a_col_0 as rule_a_col_0, "b".b_col_0 as rule_a_col_1
FROM "a", "b"
WHERE "b".b_col_1 = "a".a_col_0;
SELECT * FROM rule_a;

CREATE OR REPLACE TEMPORARY VIEW rule_b AS
SELECT DISTINCT "a".a_col_0 as rule_b_col_0, "b".b_col_0 as rule_b_col_1
FROM "a", "rule_a", "b"
WHERE "b".b_col_1 = "a".a_col_0 AND "rule_a".rule_a_col_0 = "a".a_col_0
AND "rule_a".rule_a_col_0 = "b".b_col_1
AND "rule_a".rule_a_col_1 = "b".b_col_0;
SELECT * FROM rule_b;
 \end{minted}
\caption{The same generated rules as in \ref{listing:rules:1}, but in SQL format. To enforce the
grounding, there are the SELECT queries after the rule (view) creation queries.}
\label{listing:rules:2}
\end{listing}




\section[something]{Grounders}
Each selected combination of parameters was evaluated on multiple grounders to compare the speed of
grounding of each grounder. Each grounder is described in the following sections.

\textbf{TODO:} define how to use, input and output format

\subsection{Naive bottom-up grounder}
This grounder works on a simple idea and servers as a basis for other more advanced grounders. The
process of grounding works from the most simple rules, where all variables can be unified from the
set of already grounded rules and facts and adds them to this while working it's way in the
hierarchy of the rule-template (hence the bottom-up). The pseudo-algorithm can be

\subsection{Prolog}

\subsection{LParse}

\subsection{Dlv}

\subsection{Gringo}

\subsection{PostgreSQL}


% Remove or comment out the next two lines if you are not using bibtex.
\bibliography{bib}

\end{document}


